\apendice{Especificación de Requisitos}

\section{Introducción}
Este anexo recoge los objetivos generales y la especificación de requisitos del proyecto.

\section{Objetivos generales}
El proyecto persigue los siguientes objetivos generales:
\begin{itemize}
\tightlist
\item Comprender los sistemas de recomendación tanto clásicos como basados en aprendizaje profundo.
\item Recoger y evaluar los resultados obtenidos por los dos modelos sobre diferentes conjuntos de datos.
\item Comparar los resultados.
\end{itemize}

\section{Catalogo de requisitos}
Los requisitos derivados de los objetivos del proyecto son los siguientes:
\subsection{Requisitos funcionales}
\begin{itemize}
\tightlist
\item \textbf{RF-1 Gestión de usuarios:} el programa tiene que ser capaz de gestionar los nuevos usuarios:
\begin{itemize}
\tightlist
\item \textbf{RF-1.1 Añadir usuarios:} el programa tiene que ser capaz de añadir las valoraciones de nuevos usuarios.
\end{itemize}
\item \textbf{RF-2 Gestión de los datos:} el programa tiene que ser capaz de gestionar los datos:
\begin{itemize}
\tightlist
\item \textbf{RF-2.1 Guardar datos:} el programa tiene que ser capaz de guardar los datos intermedios generados por los modelos para ahorrar tiempo en siguientes ejecuciones.
\item \textbf{RF-2.2 Cargar datos:} el programa tiene que ser capaz de cargar el conjunto de datos que el usuario quiera.
\end{itemize}
\item \textbf{RF-3 Gestión de los resultados:} el programa tiene que ser capaz de gestionar los resultados:
\begin{itemize}
\tightlist
\item \textbf{RF-3.1 Guardar resultados:} el programa tiene que ser capaz de guardar los resultados generados por los modelos.
\item \textbf{RF-3.1 Comparar resultados:} el programa tiene que ser capaz de comparar los resultados obtenidos por los distintos modelos.
\end{itemize}
\item \textbf{RF-4 Gestión de los modelos:} el programa tiene que ser capaz de gestionar los modelos:
\begin{itemize}
\tightlist
\item \textbf{RF-4.1 Guardar modelos:} el programa tiene que ser capaz de guardar los modelos generados para ahorrar tiempo en siguientes ejecuciones.
\item \textbf{RF-4.2 Cargar modelos:} el programa tiene que ser capaz de cargar el modelos seleccionado por el usuario.
\end{itemize}
\item \textbf{RF-5 Ayuda de la aplicación:} el usuario debe poder obtener ayuda sobre las funcionalidades del programa.
\end{itemize}

\subsection{Requisitos no funcionales}
\begin{itemize}
\tightlist
\item \textbf{RNF-1 Usabilidad:} la interfaz gráfica tiene que ser intuitiva y fácil de usar.
\item \textbf{RNF-2 Soporte:} el programa tiene que dar soporte a versiones iguales o mayores a Python 3.
\item \textbf{RNF-3 Localización:} el programa tiene que estar preparado para soportar varios idiomas.
\end{itemize}

\section{Especificación de requisitos}


