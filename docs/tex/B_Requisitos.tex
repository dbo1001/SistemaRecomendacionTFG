\apendice{Especificación de Requisitos}

\section{Introducción}
Este anexo recoge los objetivos generales y la especificación de requisitos del proyecto.

\section{Objetivos generales}
El proyecto persigue los siguientes objetivos generales:
\begin{itemize}
\tightlist
\item Comprender los sistemas de recomendación tanto clásicos como basados en aprendizaje profundo.
\item Recoger y evaluar los resultados obtenidos por los dos modelos sobre diferentes conjuntos de datos.
\item Comparar los resultados.
\end{itemize}

\section{Catalogo de requisitos}
Los requisitos derivados de los objetivos del proyecto son los siguientes:
\subsection{Requisitos funcionales}
\begin{itemize}
\tightlist
\item \textbf{RF-1 Gestión de los datos intermedios:} el programa tiene que ser capaz de gestionar los datos intermedios:
\begin{itemize}
\tightlist
\item \textbf{RF-1.1 Guardar matrices:} el programa tiene que ser capaz de guardar las matrices generadas por los modelos para ahorrar tiempo en siguientes ejecuciones.
\item \textbf{RF-1.2 Cargar matrices:} el programa tiene que ser capaz de cargar el conjunto de datos que el usuario quiera.
\end{itemize}
\item \textbf{RF-2 Gestión de los modelos:} el programa tiene que ser capaz de gestionar los modelos:
\begin{itemize}
\tightlist
\item \textbf{RF-2.1 Guardar modelos:} el programa tiene que ser capaz de guardar los modelos obtenidos para ahorrar tiempo en futuras ejecuciones.
\item \textbf{RF-2.2 Cargar modelos:} el programa tiene que ser capaz de cargar los modelos previamente guardados.
\end{itemize}
\item \textbf{RF-3 Gestión de los resultados:} el programa tiene que ser capaz de gestionar los resultados:
\begin{itemize}
\tightlist
\item \textbf{RF-3.1 Guardar resultados:} el programa tiene que ser capaz de guardar los resultados generados por los modelos.
\item \textbf{RF-3.2 Comparar resultados:} el programa tiene que ser capaz de comparar los resultados obtenidos por los distintos modelos.
\item \textbf{RF-3.3 Mostrar resultados:} el programa tiene que ser capaz de mostrar los resultados obtenidos por los distintos modelos.
\end{itemize}
\item \textbf{RF-4 Gestión de los usuarios:} el programa tiene que ser capaz de introducir las valoraciones de nuevos usuarios:
\begin{itemize}
\tightlist
\item \textbf{RF-4.1 Añadir valoraciones:} el programa tiene que ser capaz de simular la entrada de nuevos usuarios permitiendo el añadido de nuevas valoraciones.
\end{itemize}
\item \textbf{RF-5 Gestión de los conjuntos de datos:} el programa tiene que:
\begin{itemize}
\tightlist
\item \textbf{RF-5.1 Mostrar los conjuntos de datos:} el programa tiene que ser capaz de mostrar los distintos conjuntos de datos de prueba.
\item \textbf{RF-5.2 Seleccionar el conjunto de datos:} el programa tiene que ser capaz de mostrar dejar que el usuario seleccione uno de los conjuntos de datos de prueba.
\item \textbf{RF-5.3 Añadir conjunto de datos:} el programa tiene que ser capaz de dejar que el usuario añada un conjunto de datos propio.
\end{itemize}
\item \textbf{RF-6 Ayuda de la aplicación:} el programa tiene que ofrecer información al usuario:
\begin{itemize}
\tightlist
\item \textbf{RF-6.1 Mostrar información sobre modelos:} el programa tiene que ser capaz de mostrar información sobre los distintos modelos que se pueden escoger.
\item \textbf{RF-6.2 Mostrar información sobre conjuntos de datos:} el programa tiene que ser capaz de mostrar información sobre los distintos conjuntos de datos de prueba.
\end{itemize}
\end{itemize}

\subsection{Requisitos no funcionales}
\begin{itemize}
\tightlist
\item \textbf{RNF-1 Usabilidad:} la interfaz gráfica tiene que ser intuitiva y fácil de usar.
\item \textbf{RNF-2 Soporte:} el programa tiene que dar soporte a versiones iguales o mayores a Python 3.
\item \textbf{RNF-3 Localización:} el programa tiene que estar preparado para soportar varios idiomas.
\end{itemize}

\section{Especificación de requisitos}
En esta sección se mostrará el diagrama de casos de uso y se desarrollará cada uno de ellos.

\subsection{Diagrama de casos de uso}
\imagen{casos_de_uso}{Diagrama de casos de uso}

\subsection{Actores}
Con la aplicación solo interactuará un actor, el usuario que esté probando la aplicación en un momento determinado.

\subsection{Casos de uso}
A continuación, se desarrollará cada caso de uso: