\apendice{Documentación de usuario}

\section{Introducción}
En este apéndice se explica cómo el usuario debe instalar y utilizar la aplicación. También se indican los requisitos que necesita el usuario para llevarlo a cabo.

\section{Requisitos de usuarios}
\begin{itemize}
\tightlist
\item Navegador web.
\item Conjuntos de datos, en caso de que se quieran utilizar otros además de los de prueba.
\end{itemize}

\section{Instalación}
Como no se ha podido desplegar el proyecto con \textit{Heroku} y \textit{Gunicorn}, la aplicación no puede salir a internet. Si un usuario quiere comprobar el funcionamiento de los sistemas, tendrá que descargar e instalar \textit{Flask} para poder acceder a la aplicación desde \textit{localhost}.

Para instalar \textit{Flask} basta con ejecutar \textbf{pip install -r requirements.txt}. Esto instalará todas las dependencias que se necesiten para ejecutar el proyecto, no solo \textit{Flask}. 

Antes es necesario descargar el proyecto desde el \href{https://github.com/rnc0011/SistemaRecomendacionTFG}{repositorio} en el que se encuentra. Una vez descargado habría que navegar hasta la carpeta raíz para poder ejecutar el comando anterior.

\imagen{repositorio}{Descargar proyecto}

\imagen{navegacion_proyecto}{Navegación hasta \textit{src}}

\imagen{ejecucion_proyecto}{Ejecución del proyecto}

Metiendo la opción \textit{2} se ejecuta la interfaz web obtenida gracias a \textit{Flask}. Se tiene que abrir el navegador e ir a la dirección: \textbf{http://localhost:5000/home}.

\imagen{interfaz_web}{Interfaz web de la aplicación}

\section{Manual del usuario}


