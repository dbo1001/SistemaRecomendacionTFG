\capitulo{1}{Introducción}
Los sistemas de recomendación son herramientas fundamentales de las cuales se aprovechan las principales compañías que ofrecen servicios para proveer a los usuarios de ítems relevantes.

Compañías como \textit{Amazon}, \textit{Netflix}, \textit{Spotify} y \textit{YouTube} utilizan la información de sus usuarios y de sus productos; así como las valoraciones que los propios usuarios hacen sobre los productos que consumen con el fin de crear modelos de recomendación capaces de ofrecer a estos usuarios ítems que les serán de interés la próxima vez que interactúen con el servicio.

Tal es la importancia de los sistemas de recomendación que, por ejemplo, \textit{Netflix}, \textit{Spotify} y \textit{Trivago} organizan competiciones en las que piden mejorar sus propios sistemas \cite{wiki:Netflix_Prize}, \cite{spotify}, \cite{trivago}. El premio que ofreció \textit{Netflix} por mejorar la capacidad de predicción del modelo que se empleaba en aquel momento un 10\% fue de 1.000.000 de dólares.

En este proyecto serán objeto de estudio los modelos de recomendación clásicos y los modelos de recomendación basados en aprendizaje profundo. Así pues, se evaluará cada modelo siguiendo unas métricas comunes a los dos para poder compararles. Debido a la gran cantidad de datos que se generan actualmente, es interesante aplicar técnicas de aprendizaje profundo en los sistemas de recomendación con el fin de intentar obtener recomendaciones mejores que con los sistemas clásicos y, en la medida de lo posible, en menor tiempo.

\subsection{Estructura de la memoria}\label{estructura-memoria}
La memoria se divide en:
\begin{itemize}
\tightlist
\item \textbf{Introducción:} en este apartado se desarrolla de manera breve el tema que se va a tratar en el proyecto, así como la estructura del propio proyecto y los materiales entregados.
\item \textbf{Objetivos del proyecto:} sección en la que se indican los objetivos que se persiguen con la realización del proyecto, tanto técnicos como personales.
\item \textbf{Conceptos teóricos:} apartado en el que se explica todo lo necesario para el correcto entendimiento del tema tratado en el proyecto.
\item \textbf{Técnicas y herramientas:} sección en la que se listan todas las herramientas usadas en el proyecto, así como una breve justificación de su uso en favor de otras herramientas existentes.
\item \textbf{Aspectos relevantes del desarrollo del proyecto:} apartado en el que se explican temas de especial importancia.
\item \textbf{Trabajos relacionados:} sección en la que se indican y se desarrollan brevemente tanto artículos como proyectos que están directamente relacionados con este proyecto.
\item \textbf{Conclusiones y Líneas de trabajo futuras:} apartado en el que se recogen las conclusiones obtenidas una vez finalizado el proyecto, y las posibles mejoras que se pueden hacer en el futuro.
\end{itemize}

\subsection{Materiales adjuntos}\label{materiales-adjuntos}
Adicionalmente, junto a la memoria también se proporcionan los siguientes anexos:
\begin{itemize}
\tightlist
\item \textbf{Plan de Proyecto Software:} en este apéndice se desarrolla la planificación temporal que se ha seguido durante la realización del proyecto y la viabilidad del mismo.
\item \textbf{Especificación de Requisitos:} en este apéndice se indican y se explican los requisitos funcionales obtenidos a partir de los objetivos generales del proyecto.
\item \textbf{Especificación de diseño:} en este apéndice se explica cómo se ha estructurado el proyecto y cómo se han manipulado los datos de entrada para poder ser usados por los sistemas.
\item \textbf{Documentación técnica de programación:} esta sección contiene todo lo relacionado con la programación y la ejecución del proyecto.
\item \textbf{Documentación de usuario:} apéndice en el que se recoge todo lo que el usuario que va a utilizar la aplicación debe saber.
\end{itemize}