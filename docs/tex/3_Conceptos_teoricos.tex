\capitulo{3}{Conceptos teóricos}

Los conceptos teóricos más importantes del proyecto son los relacionados con los sistemas de recomendación y las redes neuronales.

\section{Sistemas de recomendación}\label{sistemas-de-recomendacion}
Los sistemas de recomendación son herramientas que generan recomendaciones sobre un determinado objeto de estudio, a partir de las preferencias y opiniones dadas por los usuarios. \cite{sistemas-recomendacion}

En un sistema de recomendación hay dos clases de entidades: usuarios e ítems. Los datos están representados en una matriz de utilidad de tal manera que los usuarios son filas y los ítems columnas. Para cada par hay un valor que representa el grado de preferencia de un usuario para un ítem. Se asume que la mayoría de los valores se desconocen. Es por ello, que el objetivo de un sistema de recomendación es rellenar esos espacios en blanco.

\subsection{Tipos de sistemas de recomendación}\label{tipos-sistemas-recomendacion}
Existen dos tipos de sistemas de recomendación:
\begin{itemize}
\tightlist
\item Content-based systems
\item Collaborative filtering systems
\end{itemize}

\subsection{Content-based systems}\label{content-based-systems}
Examinan propiedades de los ítems recomendados. Solo tienen en cuenta los gustos de los usuarios.

En estos sistemas, para cada ítem hay que construir un perfil compuesto por las propiedades del mismo. Por ejemplo, si los ítem son películas, algunas de las propiedades serían género, director, fecha de estreno, actores...

\subsection{Collaborative filtering systems}\label{collaborative-filtering}
Recomiendan ítems basándose en los gustos de usuarios similares.

En este caso, en lugar de usar el vector ítem-perfil de un ítem, usamos las filas de la matriz de utilidad. Los usuarios son parecidos si sus filas se acercan de acuerdo a alguna distancia.

\section{Medidas de calidad}\label{medidas-de-calidad}
Podemos hacer uso de diferentes métricas para conocer cómo de bueno nuestro sistema de recomendación es. Las métricas elegidas son:
\begin{itemize}
\tightlist
\item Precisión \textit{k}
\item Recall \textit{k}
\item \textit{AUC} Score
\item Ranking recíproco
\end{itemize}

\subsection{Precisión \textit{k}}\label{precision-k}
La precisión \textit{k} \cite{precision_at_k} mide el número de elementos conocidos que se encuentran en las primeras k posiciones del ranking de predicciones, es decir, el porcentaje de coincidencias entre los elementos conocidos y los elementos devueltos por el modelo.

\subsection{Recall \textit{k}}\label{recall-k}
El recall \textit{k} \cite{recall_at_k} la división del número de elementos relevantes conocidos que hay en las primeras \textit{k} posiciones de la lista de predicciones entre el número de elementos relevantes conocidos en toda la lista.

\subsection{\textit{AUC} Score}\label{auc-score}
El \textit{AUC} (Area Under the Curve, Área Bajo la Curva) score \cite{auc_score} es la probabilidad de que un elemento relevante escogido aleatoriamente tenga un score mayor que un elemento no relevante escogido aleatoriamente.

\subsection{Ranking recíproco}\label{ranking-reciproco}
El ranking recíproco \cite{reciprocal_rank} es el inverso del valor más alto del ranking de predicciones.

\section{Tratamiento de los datos}\label{tratamiento-datos}
Todos los datos que he utilizado a lo largo del proyecto están en formato .csv. Lo más normal es que para cada conjunto de datos tenga los siguientes archivos:
\begin{itemize}
\tightlist
\item \textit{ratings.csv} 
\item \textit{users.csv}
\item \textit{items.csv}
\end{itemize}
La estructura de estos archivos suele ser: \textit{idUser, idItem, rating, timestamp} para \textit{ratings.csv}, \textit{idUser, name, feature1, ..., featureN} para \textit{users.csv} y i\textit{dItem, name, feature1, ..., featureN} para \textit{items.csv}.

Para poder trabajar con los datos, primero los paso a \textit{DataFrames} de \textit{pandas} \cite{dataframes}.
 
\subsection{Datasets de LightFM}\label{datasets-lightfm}
\textit{LightFM} trabaja con matrices de \textit{scipy.sparse} \cite{scipy-sparse}, por lo que hace falta convertir los \textit{DataFrames} a las matrices mencionadas. 
Para ello, \textit{LightFM} proporciona la clase \textit{Dataset} \cite{dataset-lightfm}.

Gracias a esta clase, obtengo una \textit{matriz COO} \cite{coo-matrix} para las interacciones y \textit{matrices CSR} \cite{csr-matrix} para las features de usuarios e ítems. Una vez obtengo estas matrices, ya puedo crear y entrenar los diferentes modelos con \textit{LightFM}.

\section{Imágenes}

Se pueden incluir imágenes con los comandos standard de \LaTeX, pero esta plantilla dispone de comandos propios como por ejemplo el siguiente:

\imagen{escudoInfor}{Autómata para una expresión vacía}

\section{Tablas}

Igualmente se pueden usar los comandos específicos de \LaTeX o bien usar alguno de los comandos de la plantilla.

\tablaSmall{Herramientas y tecnologías utilizadas en cada parte del proyecto}{l c c c c}{herramientasportipodeuso}
{ \multicolumn{1}{l}{Herramientas} & App AngularJS & API REST & BD & Memoria \\}{ 
HTML5 & X & & &\\
CSS3 & X & & &\\
BOOTSTRAP & X & & &\\
JavaScript & X & & &\\
AngularJS & X & & &\\
Bower & X & & &\\
PHP & & X & &\\
Karma + Jasmine & X & & &\\
Slim framework & & X & &\\
Idiorm & & X & &\\
Composer & & X & &\\
JSON & X & X & &\\
PhpStorm & X & X & &\\
MySQL & & & X &\\
PhpMyAdmin & & & X &\\
Git + BitBucket & X & X & X & X\\
Mik\TeX{} & & & & X\\
\TeX{}Maker & & & & X\\
Astah & & & & X\\
Balsamiq Mockups & X & & &\\
VersionOne & X & X & X & X\\
} 
