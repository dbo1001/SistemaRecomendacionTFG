\capitulo{3}{Conceptos teóricos}

Los conceptos teóricos más importantes del proyecto son los relacionados con los sistemas de recomendación y las redes neuronales.

\section{Sistemas de recomendación}\label{sistemas-de-recomendacion}
Los sistemas de recomendación son herramientas que generan recomendaciones sobre un determinado objeto de estudio, a partir de las preferencias y opiniones dadas por los usuarios.\cite{definicion-sistemas-recomendacion}

En un sistema de recomendación hay dos clases de entidades: usuarios e ítems. Los datos están representados en una matriz de utilidad de tal manera que los usuarios son filas y los ítems columnas. Para cada par hay un valor que representa el grado de preferencia de un usuario para un ítem. Se asume que la mayoría de los valores se desconocen. Es por ello, que el objetivo de un sistema de recomendación es rellenar esos espacios en blanco.

\subsection{Tipos de sistemas de recomendación}\label{tipos-sistemas-recomendacion}
Existen dos tipos de sistemas de recomendación:
\begin{itemize}
\tightlist
\item Content-based systems
\item Collaborative filtering systems
\end{itemize}

\subsection{Content-based systems}\label{content-based-systems}
Examinan propiedades de los ítems recomendados. Solo tienen en cuenta los gustos de los usuarios.

En estos sistemas, para cada ítem hay que construir un perfil compuesto por las propiedades del mismo. Por ejemplo, si los ítem son películas, algunas de las propiedades serían género, director, fecha de estreno, actores...

\subsection{Collaborative filtering systems}\label{collaborative-filtering}
Recomiendan ítems basándose en los gustos de usuarios similares.

En este caso, en lugar de usar el vector ítem-perfil de un ítem, usamos las filas de la matriz de utilidad. Los usuarios son parecidos si sus filas se acercan de acuerdo a alguna distancia.

\section{Medidas de calidad}\label{medidas-de-calidad}
Podemos hacer uso de diferentes métricas para conocer cómo de bueno nuestro sistema de recomendación es. Las métricas elegidas son \cite{Ge10beyondaccuracy:}:
\begin{itemize}
\tightlist
\item Cobertura
\item Serendipia
\end{itemize}

\subsection{Cobertura}\label{cobertura}
La cobertura o coverage es el dominio de ítems sobre los que el sistema de recomendación puede recomendar. Podemos distinguir dos tipos de cobertura: la cobertura de predicción (porcentaje de ítems sobre los que se puede recomendar) y la cobertura de catálogo (porcentaje de ítems que son recomendados).

La que vamos a utilizar es la primera, y su fórmula es:
\begin{equation}
    \text{PC} = \dfrac{|Ip|}{|I|}
\end{equation}
donde \textit{Ip} es el conjunto de ítems sobre los que se puede hacer una predicción e \textit{I} es el conjunto de todos los ítems disponibles.

\subsection{Serendipia}\label{serendipia}
La serendipia o serendipity es una medida referente a si una recomendación es atractiva y sorprendente para el usuario. Para calcularla, primero tenemos que saber cuáles son las recomendaciones que sorprenden al usuario. Esto se calcula de la siguiente manera:
\begin{equation}
    \text{UNEXP} = \text{RS}\setminus \text{PM}
\end{equation}
donde \textit{RS} es una recomendación por nuestro sistema de recomendación y \textit{PM} es una recomendación de un modelo primitivo. Una vez hecho esto, ya podemos calcular la serendipia como:
\begin{equation}
    \text{SRDP} = \dfrac{\displaystyle\sum_{i=1}^{N}u(RSi)}{N}
\end{equation}
donde \textit{RSi} es un elemento de \textit{UNEXP} y \textit{N} es la longitud de \textit{UNEXP}. Si \textit{u(RSi)} es igual a 1, entonces la recomendación es atractiva; y si es 0, la recomendación no es atractiva y no vale.

\section{Imágenes}

Se pueden incluir imágenes con los comandos standard de \LaTeX, pero esta plantilla dispone de comandos propios como por ejemplo el siguiente:

\imagen{escudoInfor}{Autómata para una expresión vacía}

\section{Tablas}

Igualmente se pueden usar los comandos específicos de \LaTeX o bien usar alguno de los comandos de la plantilla.

\tablaSmall{Herramientas y tecnologías utilizadas en cada parte del proyecto}{l c c c c}{herramientasportipodeuso}
{ \multicolumn{1}{l}{Herramientas} & App AngularJS & API REST & BD & Memoria \\}{ 
HTML5 & X & & &\\
CSS3 & X & & &\\
BOOTSTRAP & X & & &\\
JavaScript & X & & &\\
AngularJS & X & & &\\
Bower & X & & &\\
PHP & & X & &\\
Karma + Jasmine & X & & &\\
Slim framework & & X & &\\
Idiorm & & X & &\\
Composer & & X & &\\
JSON & X & X & &\\
PhpStorm & X & X & &\\
MySQL & & & X &\\
PhpMyAdmin & & & X &\\
Git + BitBucket & X & X & X & X\\
Mik\TeX{} & & & & X\\
\TeX{}Maker & & & & X\\
Astah & & & & X\\
Balsamiq Mockups & X & & &\\
VersionOne & X & X & X & X\\
} 
