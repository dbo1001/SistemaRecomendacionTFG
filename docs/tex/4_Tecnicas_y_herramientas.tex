\capitulo{4}{Técnicas y herramientas}

\section{Metodologías}\label{metodologias}
\subsection{Scrum}\label{scrum}
Scrum es una técnica de desarrollo ágil que se caracteriza por: \cite{wiki:Scrum_(desarrollo_de_software)}
\begin{itemize}
\tightlist
\item Adoptar una estrategia de desarrollo incremental, en lugar de la planificación y ejecución completa del producto.
\item Basar la calidad del resultado más en el conocimiento tácito de las personas en equipos auto organizados, que en la calidad de los procesos completados.
\item Solapamiento de las diferentes fases del desarrollo, en lugar de realizar una tras otra en un ciclo secuencia o en cascada.
\item Realizar a diario una reunión con el objetivo de obtener realimentación sobre las tareas del equipo y los obstáculos que se presentan.
\end{itemize}
Al ser un proyecto educativo en el que solo ha habido un programador no ha hecho falta llevar a cabo las reuniones diarias. Tan solo se han realizado las reuniones de fin de sprint, que han tenido lugar cada dos semanas aproximadamente.

\section{Patrones de diseño}\label{patrones-diseno}
\subsection{MCV}\label{mvc}
Para el desarrollo del código se optó por seguir el patrón \textit{Modelo Vista Controlador}.

Según este patrón, el código se divide en tres partes principales:
\begin{itemize}
\tightlist
\item \textbf{Modelo:} contiene los archivos dedicados a manipular los datos.
\item \textbf{Vista:} contiene los archivos con los que el usuario interactúa para usar la aplicación.
\item \textbf{Controlador:} recibe las peticiones hechas desde la vista y pide los datos al modelo.
\end{itemize}
\imagen{mvc}{Esquema del patrón MVC \cite{mvc}}

\section{Control de versiones}\label{control-de-versiones}
\subsection{Git}\label{git}
\href{https://git-scm.com/}{Git} es un software de control de versiones distribuido diseñado por el creador del kernel Linux Linus Torvalds. Su objetivo es mantener un registro de los cambios que se producen en los archivos de proyectos que normalmente están compartidos.

\section{Repositorio}\label{repositorio}
\begin{itemize}
\tightlist
\item Herramientas consideradas: 
	\href{https://github.com/}{GitHub}, 
	\href{https://about.gitlab.com/}{GitLab} y
	\href{https://bitbucket.org/}{Bitbucket}
\item Herramienta escogida:
	\href{https://github.com/}{GitHub}
\end{itemize}
Se escoge \textit{GitHub} por estar familiarizado con la plataforma al haber sido usada durante toda la carrera. Junto con \textit{GitHub} se ha trabajado con \textit{GitHub Desktop} para la realización de los commits.

\section{Gestión del proyecto}\label{gestion-del-proyecto}
\begin{itemize}
\tightlist
\item Herramientas consideradas: 
	\href{https://www.zenhub.com/}{ZenHub} y 
	\href{https://trello.com/}{Trello} y
\item Herramienta escogida:
	\href{https://www.zenhub.com/}{ZenHub}
\end{itemize}
La integración de \textit{ZenHub} con \textit{GitHub} es muy superior a la de \textit{Trello}. No solo es integra directamente en la página del repositorio en \textit{GitHub} sino que además ofrece herramientas muy útiles, como los reportes, especialmente el de \textit{burndown}. Además, gracias a la extensión de \textit{ZenHub} para navegador podemos ver el estado del repositorio de una manera muy sencilla y rápida.

\section{Entorno de Desarrollo Integrado (IDE)}\label{ide}
\subsection{Python}\label{python}
\begin{itemize}
\tightlist
\item Herramientas consideradas: 
	\href{https://jupyter.org/}{Jupyter}, 
	\href{https://www.spyder-ide.org/}{Spyder}, 
	\href{https://www.jetbrains.com/pycharm/}{PyCharm} y
	\href{https://www.sublimetext.com/}{Sublime Text}
\item Herramientas escogidas:
	\href{https://jupyter.org/}{Jupyter},
	\href{https://www.spyder-ide.org/}{Spyder} y 
	\href{https://www.sublimetext.com/}{Sublime Text}
\end{itemize}
Durante el inicio del proyecto se utilizó \textit{Jupyter} para obtener en sus \textit{notebooks} modelos de recomendación iniciales sin prestar mucho detalle a la organización del proyecto y sus paquetes.

Una vez se tuvo una idea clara de cómo iba a estar estructurado el proyecto se pasó a \textit{Spyder}, con el que podíamos tener el código y la consola en el mismo sitio.

Finalmente, se pasó a \textit{Sublime Text} para realizar los últimos cambios en el código de los modelos y para obtener la aplicación \textit{Flask}.

\subsection{Documentación}\label{documentacion}
\begin{itemize}
\tightlist
\item Herramientas consideradas: 
	\href{http://www.xm1math.net/texmaker/}{Texmaker} y
	\href{https://www.openoffice.org/es/}{Apache OpenOffice}
\item Herramienta escogida:
	\href{http://www.xm1math.net/texmaker/}{Texmaker}
\end{itemize}
Se elige \textit{Texmaker} por encima de \textit{OpenOffice} debido a la novedad que supone utilizar \LaTeX{} \cite{wiki:latex} para crear documentos. 

Además, es una herramienta muy potente gracias a los comandos y funciones que tiene, que en ocasiones hacen que sea mucho más fácil realizar una tarea con \textit{Texmaker} que con \textit{OpenOffice}.

\section{Librerías}\label{librerias}
\subsection{NumPy}\label{numpy}
\href{http://www.numpy.org/}{NumPy} es una librería muy utilizada en el ámbito de la computación científica gracias a su potente manejo de arrays y a las funciones de álgebra lineal que contiene. Forma parte del ecosistema de \textit{SciPy}.

En el proyecto se emplea \textit{NumPy} para transformar listas de Python en arrays que puedan ser utilizados por los sistemas.

\subsection{pandas}\label{pandas}
\href{http://pandas.pydata.org/}{pandas}, al igual que \textit{NumPy}, es una librería muy utilizada en el campo de la computación científica debido a la facilidad que ofrece para manejar datos en tablas y series. También forma parte del ecosistema de \textit{SciPy}.

En el proyecto se emplea \textit{pandas} para el manejo inicial de los conjuntos de datos, pasando los \textit{.csv} a \textit{dataframes} \cite{dataframes}.

\subsection{LightFM}\label{lightfm}
\href{https://github.com/lyst/lightfm}{LightFM} es una implementación para Python de populares algoritmos clásicos de recomendación \cite{kulalightfm}.

Se decidió emplear \textit{LightFM} debido a su aparente sencillez y por ser una librería bastante completa. También se escogió debido a que no es un proyecto que haya pasado desapercibido, hay una buena cantidad de artículos en los que se menciona.

También se estudió la posibilidad de utilizar la librería \textit{Crab}, pero no se pudo llegar a instalar. Además, el repositorio llevaba mucho años sin actualizarse.

\subsection{Spotlight}\label{spotlight}
\href{https://github.com/maciejkula/spotlight}{Spotlight} es una librería creada por la misma gente que \textit{LightFM} que utiliza \textit{PyTorch} para construir modelos de recomendación \cite{kula2017spotlight} basados en aprendizaje profundo.

Al principio se estudió utilizar la librería de \textit{fast.ai} para obtener los modelos de aprendizaje profundo. La idea se descartó al descubrir que la misma gente que había creado \textit{LightFM} tenía una librería similiar para deep learning. Así pues, se decidió utilizar \textit{Spotlight} al suponer que el cálculo de las métricas iba a resultar muy parecido entre ambas herramientas.

\subsection{Flask}\label{flask}
\href{http://flask.pocoo.org/}{Flask} es un framework para Python con el que se obtiene una interfaz web en la que poder interactuar con el proyecto.

\subsection{tkinter}\label{tkinter}
\href{https://docs.python.org/3/library/tk.html}{Tkinter} es un módulo que forma parte de Python con el que podemos construir GUIs \cite{tkinter}.

Se utiliza en el proyecto para guardar y seleccionar las matrices generadas por los modelos y los propios modelos.

\subsection{Chardet}\label{chardet}
\href{https://pypi.org/project/chardet/}{Chardet} es un detector de encoding universal \cite{chardet}.

Se puede emplear en el proyecto para obtener el encoding de los archivos \textit{.csv}. Con ello, se logra que el usuario no necesite saber qué encoding emplea los archivos de datos que quiere utilizar.

\subsection{pickle}\label{pickle}
\href{https://docs.python.org/3/library/pickle.html}{pickle} es un módulo utilizado para la serialización de objetos en Python \cite{pickle}.

Se utiliza en el proyecto para guardar las matrices y los modelos obtenidos por los sistemas.

\section{Datasets}\label{datasets}
\subsection{Anime}\label{anime}
Este conjunto de datos \cite{CopperUnion2017} contiene información relativa a las preferencias de 73.516 usuarios sobre 12.294 películas o series de anime. Cada usuario ha valorado unos 106 animes.

El conjunto de datos se encuentra disponible en: \href{https://www.kaggle.com/CooperUnion/anime-recommendations-database/downloads/anime-recommendations-database.zip/1}{Anime Dataset}.

\subsection{Book-Crossing}\label{book-crossing}
Este conjunto de datos \cite{Ziegler2004} contiene 1.149.780 valoraciones de 278.858 usuarios sobre 271.379 libros. Hay alrededor de 11 valoraciones por usuario.

El conjunto de datos se encuentra disponible en \href{http://www2.informatik.uni-freiburg.de/~cziegler/BX/BX-CSV-Dump.zip}{Book-Crossing Dataset}.

\subsection{MovieLens}\label{movielens}
Este conjunto de datos \cite{GroupLens1998} contiene 100.000 valoraciones de 943 usuarios sobre 1.682 películas. Hay unas 106 valoraciones por usuario.

El conjunto de datos se encuentra disponible en los propios archivos descargados por la librería \textit{LightFM} \cite{DBLP:conf/recsys/Kula15}.

\subsection{Last.FM}\label{last.fm}
Este conjunto de datos \cite{Cantador:RecSys2011} contiene las veces que 1.892 usuarios han escuchado a 17.632 artistas. Cada usuario ha escuchado alrededor de unos 49 artistas.

El conjunto de datos se encuentra disponible en \href{http://files.grouplens.org/datasets/hetrec2011/hetrec2011-lastfm-2k.zip}{Last.FM}.

\subsection{Dating Agency}\label{dating-agency}
Este conjunto de datos \cite{brozovsky07recommender} contiene 17.359.346 valoraciones de 135.359 usuarios sobre 168.791 perfiles de otros usuarios. Hay unas 128 valoraciones por usuario.

El conjunto de datos se encuentra disponible en \href{http://www.occamslab.com/petricek/data/libimseti-complete.zip}{Dating Agency}.

\imagen{usuarios_datasets}{Número de usuarios por cada dataset de prueba}

\imagen{items_datasets}{Número de ítems por cada dataset de prueba}

\imagen{valoraciones_usuario_datasets}{Número de valoraciones por usuario de cada dataset de prueba}