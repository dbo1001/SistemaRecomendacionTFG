\capitulo{4}{Técnicas y herramientas}

\section{Metodologías}\label{metodologias}
\subsection{Scrum}\label{scrum}
Descripción

\section{Control de versiones}\label{control-de-versiones}
\subsection{Git}\label{git}
Descripción

\section{Repositorio}\label{repositorio}
\subsection{GitHub}\label{github}
Descripción

\section{Gestión del proyecto}\label{gestion-del-proyecto}
\subsection{ZenHub}\label{zenhub}
Descripción

\section{Entorno de Desarrollo Integrado (IDE)}\label{ide}
\subsection{Python}\label{python}
\begin{itemize}
\tightlist
\item Herramientas consideradas: 
	\href{https://jupyter.org/}{Jupyter}, 
	\href{https://www.spyder-ide.org/}{Spyder} y
	\href{https://www.jetbrains.com/pycharm/}{PyCharm}
\item Herramienta escogida:
	Jupyter \href{https://jupyter.org/}{Jupyter}
\end{itemize}
Justificación

\subsection{Documentación}\label{documentacion}
\begin{itemize}
\tightlist
\item Herramientas consideradas: 
	\href{http://www.xm1math.net/texmaker/}{Texmaker} y
	\href{https://www.openoffice.org/es/}{Apache OpenOffice}
\item Herramienta escogida:
	\href{http://www.xm1math.net/texmaker/}{Texmaker}
\end{itemize}
Justificación

\section{Librerías}\label{librerias}
\subsection{NumPy}\label{numpy}
\href{http://www.numpy.org/}{NumPy} descripción
\subsection{SciPy}\label{scipy}
\href{https://www.scipy.org/scipylib/index.html}{SciPy} descripción
\subsection{LightFM}\label{lightfm}
\href{https://github.com/lyst/lightfm}{LightFM} descripción

