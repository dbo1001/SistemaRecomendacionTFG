\apendice{Especificación de diseño}

\section{Introducción}
En este apéndice se explica cómo están conformados los datos que utilizan las librerías usadas en el proyecto, así como la forma en la que está estructurado el mismo.

\section{Diseño de datos}
Todos los datos que se han utilizado a lo largo del proyecto están en formato .csv. Lo más normal es que para cada conjunto de datos se tengan los siguientes archivos:
\begin{itemize}
\tightlist
\item \textit{ratings.csv} 
\item \textit{users.csv}
\item \textit{items.csv}
\end{itemize}
La estructura de estos archivos suele ser: \textit{idUser, idItem, rating, timestamp} para \textit{ratings.csv}, \textit{idUser, name, feature1, ..., featureN} para \textit{users.csv} y i\textit{dItem, name, feature1, ..., featureN} para \textit{items.csv}.

\tablaSinColores{Formato ratings.csv}{l l l l}{4}{ratings}
{\textbf{Id Usuario} & \textbf{Id Ítem} & \textbf{Valoración} & \textbf{Timestamp}\\}{
	1 & 1 & 5 & 1562057352 \\
	1 & 2 & 1 & 1562057354 \\
	2 & 120 & 4 & 1562057370 \\
	. & . & . & . \\
	. & . & . & . \\
	. & . & . & . \\
	Usuario N & Ítem N & Valoración & timestamp \\
}

\tablaSinColores{Formato users.csv}{l l l l l}{5}{usuarios}
{\textbf{Id Usuario} & \textbf{Nombre} & \textbf{Feature 1} & \textbf{...} & \textbf{Feature N}\\}{
	1 & Pepe & Feature 1 & ... & Feature N \\
	2 & Ana & Feature 1 & ... & Feature N \\
	3 & Luis & Feature 1 & ... & Feature N \\
	. & . & . & . & . \\
	. & . & . & . & . \\
	. & . & . & . & . \\
	Usuario N & Nombre Usuario N & Feature 1 & ... & Feature N \\
}

\tablaSinColores{Formato items.csv}{l l l l l}{5}{items}
{\textbf{Id Ítem} & \textbf{Nombre} & \textbf{Feature 1} & \textbf{...} & \textbf{Feature N}\\}{
	1 & Star Wars Episodio IV & Feature 1 & ... & Feature N \\
	2 & Spotlight & Feature 1 & ... & Feature N \\
	3 & Seven & Feature 1 & ... & Feature N \\
	. & . & . & . & . \\
	. & . & . & . & . \\
	. & . & . & . & . \\
	Ítem N & Nombre Ítem N & Feature 1 & ... & Feature N \\
}

Para poder trabajar con los datos primero se pasan a \textit{DataFrames} de \textit{pandas} \cite{dataframes}.

\subsection{Datos con LightFM}\label{datos-lightfm}
Una vez obtenidos los \textit{DataFrames} para cada \textit{.csv}, es necesario convertirlos a la clase \textit{Dataset} de \textit{LightFM} \cite{dataset-lightfm} para poder trabajar con ellos. 

Esta clase se encarga de convertir los datos almacenados en los \textit{DataFrames} en \textit{matrices COO} y \textit{matrices CSR}.

\subsection{Datos con Spotlight}\label{datos-spotlight}
A diferencia de \textit{LightFM}, en \textit{Spotlight} no se trabaja con matrices dispersas, si no con arrays de \textit{NumPy} (aunque se ofrece la posibilidad de transformar los arrays en \textit{matrices COO} y \textit{matrices CSR}).

\textit{Spotlight} utiliza su propia clase, \textit{Interactions} \cite{interactions-spotlight}, para convertir los \textit{DataFrames} y así poder utilizar los datos.

Otro aspecto a tener en cuenta en \textit{Spotlight} son las secuencias, utilizadas en el modelo de secuencia implícito \cite{modelo-secuencia}. Las recomendaciones pueden verse como secuencias; dados los ítems con los que ha interactuado un usuario, ¿cuáles serán los próximos ítems con los que interactuará? De esta manera, una vez obtenidas las interacciones "normales", hay que pasarlas a interacciones de secuencia con el método \textbf{to\_sequence()} \cite{to_sequence}.

\subsection{Persistencia}\label{persistencia}
Para la realización del proyecto se necesita guardar de alguna manera tanto los datos intermedios (matrices de interacción) como los propios modelos y los resultados obtenidos al evaluarlos.

Esto se consigue gracias a \textit{pickle} \cite{pickle}.

\section{Diseño procedimental}
En este apartado se utilizan diagramas de secuencia para explicar el funcionamiento de la aplicación.

\imagen{secuencia_obtener_modelo}{Diagrama de secuencia para obtener un nuevo modelo}

\imagen{secuencia_obtener_metricas}{Diagrama de secuencia para obtener métricas y predicciones}

\imagen{secuencia_add_valoraciones}{Diagrama de secuencia para añadir valoraciones}

\section{Diseño arquitectónico}
Para la realización de este proyecto se ha seguido el patrón arquitectónico MVC (\textit{Modelo Vista Controlador}). El objetivo de este patrón es dividir el código en función de su propósito. Sus partes son:
\begin{itemize}
\tightlist
\item \textit{Modelo}: el acceso a los datos. Se corresponde con las clases de Entrada, Salida y Persistencia; que leen los datos para dárselo al sistema de recomendación y guardan los resultados.
\item \textit{Vista}: la visualización de los datos. Se corresponde con las clases de Interfaz, Flask y Forms, que muestran la información solicitada.
\item \textit{Controlador}: la manipulación de los datos. Se corresponde con los clases de Sistema, que crean los sistemas de recomendación.
\end{itemize}

La estructura del proyecto siguiendo este patrón quedaría de la siguiente forma:
\imagen{diagrama_src}{Diagrama UML del proyecto}
Por separado, los paquetes contienen:
\imagen{diagrama_vista}{Diagrama UML del paquete \textit{vista}}
\imagenGrande{diagrama_controlador}{Diagrama UML del paquete \textit{controlador}}{1.0}
\imagenGrande{diagrama_modelo}{Diagrama UML del paquete \textit{modelo}}{1.25}

\section{Diseño de la interfaz}\label{diseño-gui}
A continuación se muestra el diseño que tendrá la interfaz web (es posible que la versión final cambie).

\begin{landscape}
\imagenGrande{wireframe}{Interaz web}{1.75}
\end{landscape} 
