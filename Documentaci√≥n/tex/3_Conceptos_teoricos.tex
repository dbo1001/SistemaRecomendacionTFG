\capitulo{3}{Conceptos teóricos}

Los conceptos teóricos más importantes del proyecto son los relacionados con los sistemas de recomendación y las redes neuronales.

\section{Sistemas de recomendación}\label{sistemas-de-recomendacion}
Los sistemas de recomendación son herramientas que generan recomendaciones sobre un determinado objeto de estudio, a partir de las preferencias y opiniones dadas por los usuarios. \cite{sistemas-recomendacion}

En un sistema de recomendación hay dos clases de entidades: usuarios e ítems. Los datos están representados en una matriz de utilidad de tal manera que los usuarios son filas y los ítems columnas. Para cada par hay un valor que representa el grado de preferencia de un usuario para un ítem. Se asume que la mayoría de los valores se desconocen. Es por ello, que el objetivo de un sistema de recomendación es rellenar esos espacios en blanco.

Existen tres tipos de sistemas (o modelos) de recomendación:
\begin{itemize}
\tightlist
\item Modelos basados en contenido
\item Modelos colaborativos
\item Modelos híbridos
\end{itemize}

Los tres tipos de modelos se han podido obtener con \textit{LightFM}, no así con \textit{Spotlight}, con el cual no se puede obtener un modelo híbrido.

\subsection{Modelos basados en contenido}\label{content-based-systems}
Este tipo de sistemas tienen en cuenta los gustos del usuario y las características de los ítems.

Para cada ítem hay que construir un perfil compuesto por las propiedades del mismo. Por ejemplo, si los ítem son películas, algunas de las propiedades serían género, director, fecha de estreno, reparto, etc. Una vez hecho esto, se recomiendan ítems cuyas propiedades sean parecidas a los ítems que el usuario ha valorado positivamente.

En \textit{Spotlight}, este tipo de sistema se correspondería con el modelo de factorización implícito y con el modelo de secuencia implícito. En estos modelos se utiliza la matriz de interacción entre usuarios e ítems pero sin las valoraciones.

\subsection{Modelos colaborativos}\label{collaborative-filtering}
Recomiendan ítems basándose únicamente en los gustos de usuarios similares.

En este caso, en lugar de usar el vector ítem-perfil de un ítem, usamos las filas de la matriz de utilidad. Los usuarios son parecidos si sus filas se acercan de acuerdo a alguna distancia.

En \textit{Spotlight}, este tipo de sistema se correspondería con el modelo de factorización explícito.

\subsection{Modelos híbridos}\label{modelos-hibridos}
Este tipo de sistemas utilizan las dos técnicas anteriores para ofrecer mejores recomendaciones.

Así pues, se recomendarán ítems que sean parecidos a los ítems valorados positivamente por usuarios parecidos.

\section{Medidas de calidad}\label{medidas-de-calidad}
Podemos hacer uso de diferentes métricas para conocer cómo de bueno es nuestro sistema de recomendación. Dado que las dos librerías que se han utilizado en el proyecto las ha creado la misma persona, se utilizarán las métricas ofrecidas en ellas por su gran similitud. Las métricas son:
\begin{itemize}
\tightlist
\item Precisión \textit{k}
\item Recall \textit{k}
\item Score \textit{AUC}
\item Ranking recíproco
\item \textit{Root Mean Square Error}
\end{itemize}

\subsection{Precisión \textit{k}}\label{precision-k}
La precisión \textit{k} \cite{precision_at_k} mide el número de elementos relevantes conocidos que se encuentran en las primeras \textit{k} posiciones del ranking de predicciones, es decir, el porcentaje de coincidencias entre los elementos relevantes conocidos y los elementos devueltos por el modelo. Su fórmula es:
\begin{equation}
Precision\;at\;k = \frac{ERC \subset kRP}{k}
\end{equation}
siendo \textit{ERC} los elementos relevantes conocidos y \textit{kRP} las primeras \textit{k} posiciones del ranking de predicciones.

\subsection{Recall \textit{k}}\label{recall-k}
El recall \textit{k} \cite{recall_at_k} la división del número de elementos relevantes que hay en las primeras \textit{k} posiciones del ranking de predicciones entre el número de elementos relevantes conocidos en el conjunto de test. Su fórmula es:
\begin{equation}
Recall\;at\;k = \frac{ERk}{ERT}
\end{equation}
siendo \textit{ERk} los elementos relevantes en las primeras \textit{k} posiciones del ranking de resultados y \textit{ERT} los elementos relevantes del conjunto de test.

\subsection{\textit{AUC} Score}\label{auc-score}
El \textit{AUC} (Area Under the Curve, Área Bajo la Curva) score \cite{auc_score} es la probabilidad de que un elemento relevante escogido aleatoriamente tenga un score mayor que un elemento no relevante escogido aleatoriamente.

\subsection{Ranking recíproco}\label{ranking-reciproco}
El ranking recíproco \cite{reciprocal_rank} es el inverso del score del elemento más relevante devuelto en el ranking de resultados. Su fórmula es:
\begin{equation}
Reciprocal\;rank = \frac{1}{SER}
\end{equation}
siendo \textit{SER} el score del elemento más relevante devuelto en el ranking de resultados.

En el modelo de \textit{Spotlight} aparece como \textit{MMR}, o \textit{Mean Reciprocal Rank}.

\subsection{Root Mean Square Error}\label{rmse}
El \textit{RMSE} se define como el error que hay entre dos conjuntos de datos. Aplicado a este caso, el \textit{RMSE} \cite{rmse} compara un valor conocido con un valor observado. Su fórmula sería la siguiente:
\begin{equation}
RMSE = \sqrt{\frac{\sum_{i=1}^n(C_{i} - O_{i})^{2}}{n}}
\end{equation}
siendo \textit{C} una recomendación conocida y \textit{O} una recomendación aportada por el sistema.

\section{Tratamiento de los datos}\label{tratamiento-datos}
Lo más común en los sistemas de recomendación es tener una gran cantidad de usuarios y de ítems. En cambio, la cantidad de valoraciones de ítems por cada usuario que se tiene es muy pequeña. Esto hace que la matriz con la que representamos las valoraciones sea muy grande pero con muchos valores a 0 (o celdas vacías, dependiendo de la representación escogida).  

Esto es lo que llamamos \textit{matrices dispersas} \cite{wiki:Sparse_matrix}. Trabajar con estas matrices supone un gran coste de rendimiento, por lo que necesitamos convertirlas a otras estructuras con las que poder trabajar mejor.

En el caso de \textit{LightFM}, las estructuras utilizadas son:
\begin{itemize}
\tightlist
\item Matrices \textit{COO}
\item Matrices \textit{CSR}  
\end{itemize}

\textit{Spotlight}, en cambio, utiliza directamente los arrays de \textit{numpy}; aunque también ofrece métodos para transformar estos arrays a matrices \textit{COO} y \textit{CSR}.

\subsection{Matrices COO}\label{matrices-coo}
Las \textit{matrices COO} \cite{coo-matrix}, o matrices coordenadas \textit{(COOrdinate)}, dividen la matriz dispersa en tres vectores:
\begin{itemize}
\tightlist
\item Vector con los valores no nulos
\item Vector con el índice de las filas
\item Vector con el índice de las columnas
\end{itemize}
Esto lo podemos ver más claro con el siguiente ejemplo. Dada la siguiente matriz dispersa
\[\begin{bmatrix}
27&0&8&14&0&0\\
0&7&0&0&0&0\\
3&0&7&0&0&9\\
0&0&0&0&0&1\\
4&0&77&0&0&0\\
\end{bmatrix}\]
obtendríamos los siguientes vectores:
\begin{itemize}
\tightlist
\item Vector de valores: \[\begin{bmatrix} 27&8&14&7&3&7&9&1&4&77\\\end{bmatrix}\]
\item Vector de filas: \[\begin{bmatrix} 1&1&1&2&3&3&3&4&5&5\\\end{bmatrix}\]
\item Vector de columnas: \[\begin{bmatrix} 1&3&4&2&1&3&6&6&1&3\\
\end{bmatrix}\]
\end{itemize}

Este tipo de estructura es utilizada por \textit{LightFM} para representar la matriz de interacciones entre usuarios e ítems.

\subsection{Matrices CSR}\label{matrices-csr}
Otras estructuras ampliamente utilizadas, por ser más compactas que las matrices \textit{COO}, son las matrices \textit{CSR}, o \textit{Compressed Sparse Row} \cite{csr-matrix}. Estas matrices no almacenan los índices de las filas, si no que almacenan punteros a los inicios de fila.

Utilizando como ejemplo la matriz dispersa del caso anterior, ahora obtendríamos:
\begin{itemize}
\tightlist
\item Vector de valores: \[\begin{bmatrix} 27&8&14&7&3&7&9&1&4&77\\\end{bmatrix}\]
\item Vector de columnas: \[\begin{bmatrix} 1&3&4&2&1&3&6&6&1&3\\
\end{bmatrix}\]
\item Vector de filas: \[\begin{bmatrix} 1&4&5&9&10&11\\\end{bmatrix}\]
\end{itemize}

Este tipo de estructura es utilizada por \textit{LightFM} para representar las matrices de features de usuarios e ítems.

\section{Paralelismo}\label{paralelismo}
Debido a la gran cantidad de datos que se pueden llegar a manejar en los sistemas de recomendación y, en general, en cualquier aplicación de \textit{Big Data}, \textit{Deep Learning}, etc., es muy interesante el uso de computación paralela.

Esto se consigue gracias a \textit{CUDA}, \textit{Compute Unified Device Architecture} \cite{cuda}. Es una plataforma de computación en paralelo desarrollada por \textit{nVidia} para ser usada en sus tarjetas gráficas.

Ya que las \textit{GPUs} tienen muchísimos más núcleos que las \textit{CPUs}, las primeras se pueden utilizar para ejecutar operaciones en paralelo. Esto hace que se ahorre mucho tiempo en campos como los sistemas de recomendación.

\imagen{cuda}{Diferencia de núcleos entre CPU y GPU}

En el proyecto se puede utilizar \textit{CUDA} con el sistema de \textit{Spotlight}.